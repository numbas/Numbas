%Copyright 2011 Newcastle University
%
%   Licensed under the Apache License, Version 2.0 (the "License");
%   you may not use this file except in compliance with the License.
%   You may obtain a copy of the License at
%
%       http://www.apache.org/licenses/LICENSE-2.0
%
%   Unless required by applicable law or agreed to in writing, software
%   distributed under the License is distributed on an "AS IS" BASIS,
%   WITHOUT WARRANTIES OR CONDITIONS OF ANY KIND, either express or implied.
%   See the License for the specific language governing permissions and
%   limitations under the License.

\begin{chapter}{\label{cha:scorm}SCORM objects}
  The \verb"numbas.py" script will, by default, produce a directory
  containing files to run an exam locally in a browser.
  
  It is also possible to compile your exam into a SCORM 2004 compliant object
  (\url{http://scorm.com}), so that it could theoretically be used within a
  VLE, such as Blackboard or Moodle.  At the time of writing, many VLEs
  (including Blackboard and Moodle) do not support the SCORM 2004 standard well
  or at all, making it difficult to integrate the \numbas\ SCORM objects into
  them.
  
  Nevertheless, the SCORM objects that \numbas\ does produce adhere to the
  standard.  When the VLEs catch up, the SCORM objects can be used.
  
  \section{MathJax}
  By default, \numbas\ will use a copy of MathJax that is available on the
  MathJax Content Delivery Network (CDN).  Part of the SCORM standard states
  that an exam object must be self-contained, and in particular, that there
  should be no dependence on a network connection.  In order to overcome this
  limitation, SCORM compliant exams must include a local copy of MathJax.
  
  \section{\label{sec:local_mathjax}Installing a local copy of MathJax}
  Follow these steps to install a local copy of MathJax into a new
  \emph{theme}.  This is necessary for producing SCORM objects, and for
  overcoming the Mozilla Firefox security issue with web fonts, as explained in
  \S\ref{sec:mathjax_fonts}.
  %
  \begin{itemize}
    \item From the top-level \numbas\ directory, change to the
      \codefile{themes} directory, where you should see a directory called
      \codefile{default}.  This is the directory which controls the default
      theme.  Copy the contents of \codefile{default} to a new directory called
      \codefile{local}.
    \item Download a copy of MathJax from \url{http://www.mathjax.org}.
      (MathJax is also available as a git repository.)  Unpack the contents
      into a directory called \codefile{MathJax}, within \codefile{local}, at
      the same level as \codefile{index.html}.  Note that this will take a good
      10 minutes on a Windows machine, due to the large number of very small
      files in the MathJax archive.
    \item Edit the file \codefile{index.html} in \codefile{local/files}, and
      change the line which reads
      %
      {\small
      \begin{Verbatim}
  <SCRIPT SRC="http://cdn.mathjax.org/mathjax/latest/MathJax.js">
      \end{Verbatim}
      }
      %
      so that it now reads
      %
      {\small
      \begin{Verbatim}
  <SCRIPT SRC="MathJax/MathJax.js">
      \end{Verbatim}
      }
  \end{itemize}
  %
  The result of the above steps is to create a new theme called
  \codefile{local}, which includes a local copy of MathJax.  Exams produced
  using this theme will have no dependence on a network connection.

  \subsection{Using the new theme}
  To use the new \codefile{local} theme, compile your exams using the
  \verb"-t" option, which takes the name of the theme as an argument.
  Therefore, to compile the example exam, you would type
  %
  \begin{Verbatim}
    python bin/numbas.py -t local exams/example.exam
  \end{Verbatim}
  %
  You can check that the resulting exam will use a local copy of MathJax by
  ensuring that the \codefile{MathJax} directory from the theme has been copied
  to \codefile{output/example}, and that \codefile{index.html} has the
  alteration stated above.

  \section{Creating a SCORM object}
  The procedure for creating a SCORM object is exactly as in the previous
  section, except you must include the \verb"-s" option, so that the
  interpreter knows to include the extra files necessary to produce a valid
  object, \eg
  %
  \begin{Verbatim}
    python bin/numbas.py -t local -s exams/example.exam
  \end{Verbatim}
  %
  Incorporating the resulting object into a VLE is beyond the scope of this
  manual.

\end{chapter}
